\chapter*{Введение}							% Заголовок
\addcontentsline{toc}{chapter}{Введение}	% Добавляем его в оглавление

\newcommand{\actuality}{}
\newcommand{\progress}{}
\newcommand{\aim}{{\textbf\aimTXT}}
\newcommand{\tasks}{\textbf{\tasksTXT}}
\newcommand{\novelty}{\textbf{\noveltyTXT}}
\newcommand{\influence}{\textbf{\influenceTXT}}
\newcommand{\methods}{\textbf{\methodsTXT}}
\newcommand{\defpositions}{\textbf{\defpositionsTXT}}
\newcommand{\reliability}{\textbf{\reliabilityTXT}}
\newcommand{\probation}{\textbf{\probationTXT}}
\newcommand{\contribution}{\textbf{\contributionTXT}}
\newcommand{\publications}{\textbf{\publicationsTXT}}



Эффекты излучения равномерно движущихся зарядов в присутствии различных сред и объектов были обнаружены и объяснены отечественными физиками в середине прошлого века. Плеяда выдающихся открытий началась с того, что П. А. Черенков обнаружил~\cite{Cherenkov34} необычное (по ряду свойств) свечение при наблюдении люминесценции солей урана, а его научный руководитель С. И. Вавилов предпринял попытку объяснить этот эффект~\cite{Vavilov34}. Первое правильное теоретическое описание было представлено И. Е. Таммом и И. М. Франком несколькими годами позже~\cite{TammFrank37}. Были получены выражения для электромагнитного поля, возбуждаемого зарядом, движущимся равномерно и прямолинейно в изотропной негиротропной среде с частотной дисперсией, а также формула для потерь энергии на излучение на единицу пройдённого зарядом пути (формула Тамма-Франка). В 1958 году П. А. Черенкову, И. Е. Тамму и И. М. Франку была присуждена Нобелевская премия по физике за открытие и истолкование излучения, названного излучением Вавилова-Черенкова~\cite{Cherenkov1959, Tamm1959, Frank1959}.

Вслед за открытием нового эффекта, его изучением занялись многие выдающиеся физики. В частности, В. Л. Гинзбургом был рассмотрен случай анизотропной среды [7] и развита квантовая теория эффекта [8]. Все те же И. Е. Тамм и И. М. Франк в другой совместной работе попытались объяснить природу эффекта на микроскопическом уровне с классической точки зрения [9]. Магнитная проницаемость среды в задачах об излучении Вавилова-Черенкова впервые была учтена в работах А. Г. Ситенко [10], а также Д. Д. Иваненко и В. Н. Цытовича [11]. За рубежом G. Collins и V. Reiling экспериментально подтвердили идеи И. Е. Тамма и И. М. Франка [12]. Знаменитый физик-теоретик E. Fermi также занимался новым явлением: он учёл поглощение в диэлектрике и изучил вопрос о влиянии поляризации среды на энергетические потери заряда [13]. В трудах O. Halpern, H. Hall и R. M. Sternheimer [14–17] были рассмотрены среды с более сложными дисперсионными свойствами, активно развивалась теория эффекта в кристаллах и других анизотропных средах [18–25]. В монографиях И. М. Франка [26], Дж. Джелли [27], В. П. Зрелова [28], обзорах Б. М. Болотовского [29,30] и ряде других изданий [31–33] представлены теоретические аспекты излучения Вавилова-Черенкова, возникающего в различных ситуациях, как в безграничных, так и в ограниченных средах, а также описано множество применений этого эффекта. Одними из самых широко используемых изобретений, основанных на излучении Вавилова-Черенкова, стали датчики скорости заряженных частиц.

Отметим, что большинство трудов по изучению эффекта Вавилова–Черенкова, ставят своей задачей нахождение энергетических характеристик излучения. В то же время, структура электромагнитного поля частицы изучалась относительно редко. В этой связи можно отметить, например, работу самого И. Е. Тамма [34], где развивался асимптотический метод определения максимума поля, работы Г. Н. Афанасьева и В. Г. Картавенко [35,33], а также А. В. Тюхтина и С. Н. Галямина [36,37].

В последние десятилетия исследования в области излучения Вавилова-Черенкова в значительной степени связаны с изучением и применением этого эффекта в волноводных структурах, содержащих диэлектрики. Впервые подобные задачи начали рассматриваться ещё в середине XX века [30]. Многие из современных теоретических и экспериментальных работ посвящены кильватерному методу ускорения заряженных частиц, в основе которого лежит идея об ускорении одного пучка в волновом поле другого, а также генерации излучения в терагерцовом диапазоне частот [38–48]. Можно отметить и новый метод определения энергии пучков заряженных частиц по частотам возбуждаемых им волноводных мод, предложенный в работах [49–53].

В последнее время уделяется также внимание излучению Вавилова-Черенкова в искусственных структурах, называемых метаматериалами. Они состоят из большого количества макроскопических объектов, которые играют роль «молекул». При условии, что длина рассматриваемых волн значительно больше расстояния между этими объектами, структуру можно рассматривать как «среду», характеризуемую определёнными макропараметрами. В первую очередь, эти исследования связаны с так называемыми «левыми» средами [54–57], в которых излучение Вавилова-Черенкова имеет «обратный» характер, то есть плотность потока энергии направлена под тупым углом к скорости движения заряда.

Излучению в присутствии проволочных метаматериалов, которое рассматривается во второй главе настоящей диссертации, посвящено лишь небольшое число работ [58–62]. Работа M. G. Silveirinha и S. I. Maslovski [58] посвящена излучению неподвижных источников: диполя, расположенного параллельно проводам, и сосредоточенного источника напряжения, присоединённого к одному из проводов (в неограниченном проволочном метаматериале). Те же авторы вместе с D. E. Fernandes рассмотрели излучение бесконечно длинной заряженной нити, движущейся в проволочном метаматериале или вдоль его границы перпендикулярно себе и проводам [59]. Следует отметить, что данная работа вышла одновременно с работой автора настоящей диссертации [105] (совместной с А.В. Тюхтиным), в которой рассматривается поле точечного заряда в проволочном метаматериале. В [59], как и в нашей работе [105] отмечается концентрация излучения вдоль определённых линий позади источника и отсутствие порогового значения скорости заряда для генерации излучения. В работе [60] T. A. Morgado, D. E. Fernandes и M. G. Silveirinha рассчитывается сила радиационного торможения, воздействующая на заряд в проволочном метаматериале. Также стоит отметить статью [61] А. В. Тюхтина и В. В. Воробьева, результаты которой не вошли в данную диссертацию. Она посвящена излучению заряда, движущегося внутри неограниченного метаматериала, который состоит из длинных проводов, покрытых диэлектрической или магнитной оболочкой. При этом скорость заряда параллельна проводам, а генерация излучения обеспечивается только за счёт наличия оболочек (в настоящей диссертации рассматриваются пучки, движущиеся перпендикулярно проводам без оболочек).

Помимо теоретических, проводились также и экспериментальные работы по детектированию излучения от зарядов, движущихся в присутствии периодических проволочных структур. В Томском политехническом университете В. В. Соболева, Г. А. Науменко и В. В. Блеко проводили измерение излучения, генерируемого при пролёте пучка электронов мимо мишеней в виде треугольной призмы и плоской сетки из параллельных периодически расположенных проводов [62]. Согласно результатам этого исследования, возбуждаемое черенковское излучение распространяется внутри призмы под углом, соответствующим аналитическим результатам, представленным в настоящей диссертации.

Вслед за открытием излучения Вавилова-Черенкова, последовало теоретическое предсказание и экспериментальное подтверждение переходного и дифракционного излучения. Эти два типа излучения тесно связаны между собой: первое возникает при пролёте заряда сквозь неоднородность, а другое — при пролёте вблизи неоднородности, когда собственное поле заряженной частицы дифрагирует на ней. Переходное излучение было теоретически предсказано В. Л. Гинзбургом и И. М. Франком в 1946 году [63]. Особенно активно переходное излучение исследовалось в Ереванском Институте Физики под руководством Г. М. Гарибяна [64–66], где оно было зафиксировано в ходе эксперимента в 1958 году. Спустя некоторое время, Г. М. Гарибян [64] и К. А. Барсуков [67] независимо показали, что энергия, излучённая ультрарелятивистскими зарядами в рентгеновском диапазоне частот, пропорциональна их собственной энергии. На основе этого были изобретены датчики энергии ультрарелятивистских частиц, основанные на регистрации переходного излучения (черенковские датчики могли регистрировать лишь скорость, а у ультрарелятивистских зарядов она почти одинакова, хотя энергия может сильно отличаться). Развитие последующей теории переходного излучения зачастую тесно связано с методами диагностики и детектирования частиц, что стимулировалось бурным развитием физики ускорителей. Подробно многие задачи о переходном излучении описаны и разобраны в ряде монографий [68–70] и обзоров [71–77].

В последние годы теория переходного излучения продолжает активно развиваться. Например, С. Н. Галямин с соавторами рассмотрел вопрос о генерации переходного излучения при прохождении заряда через границу «левой» среды [37,57], а также при влёте в анизотропную диспергирующую среду, в которой возможно обратное черенковское излучение [78]. И. П. Иванов и Д. В. Карловец в своей работе [79] предложили методику регистрации переходного излучения, генерируемого магнитным моментом. Б. М. Болотовский, А. В. Кольцов и А. В. Серов в монографии [80] исследовали переходное излучение на ряде объектов, имеющих ребра или вершины (двугранные и трёхгранные углы, коническая поверхность).

Особо стоит остановиться на исследовании излучения, генерируемого зарядом в присутствии плоской структуры из протяжённых параллельных проводников с малым периодом. Такое исследование обычно проводилось с помощью замены реальной структуры плоскостью, на которой задаются те или иные «эффективные» граничные условия. В простейшем приближении это условия идеальной проводимости в одном направлении и полной непроводимости в ортогональном направлении (модель идеально проводящего в одном направлении «экрана»). В таком приближении в работах К. А. Барсукова с соавторами рассматривалось поле точечного заряда при его движении перпендикулярно [81] и параллельно [82,83] к плоскости бесконечного «экрана». В работе К. А. Барсукова и С. Х. Бековой [87] рассматривались поверхностные волны от точечного заряда, движущегося вдоль ребра полуплоскости с идеальной проводимостью в одном направлении. Авторы рассчитали частотные спектры поля и потерь энергии на единицу длины пути для случая движения заряда по плоскости сетки, однако без анализа пространственно-временной структуры поля волны.

Влияние геометрии идеальных проводников на излучение точечного заряда в присутствии неограниченной планарной структуры рассматривалось в работах К. А. Барсукова и С. Х. Бековой [84], а также В. Н. Красильникова и А. В. Тюхтина [85]. Так, в работе [85] был проведён анализ излучения точечного заряда, пересекающего в ортогональном направлении структуру из параллельных проводников, с учётом их геометрических особенностей и конечной проводимости (при применении метода усреднённых граничных условий [86]). Однако основное внимание в отмеченных работах уделялось объёмному излучению точечного заряда, в то время как поверхностные волны, представляющие особый интерес для настоящего исследования, анализировались недостаточно, без анализа их пространственно-временной структуры. Подчеркнём, что анализ излучения от неточечных зарядов в присутствии планарных структур с малым периодом в доступной нам литературе не проводился.

Экспериментальное наблюдение излучения от зарядов на планарных структурах описано в работе В. В. Соболевой, Г. А. Науменко и В. В. Блеко [62]. Также отметим экспериментальную работу этих же авторов совместно с А. О. Шумейко по переходному излучению от проволочного метаматериала [88].
Как уже отмечалось, дифракционное излучение по своей природе сходно с переходным. При этом строгая теория дифракционного излучения более сложна с точки зрения математики. Первая работа по дифракционному излучению была проделана А. П. Казанцевым и Г. И. Сурдутовичем [89]. Позже было получено решение задачи о переходном излучении заряженной частицы, пролетающей мимо идеально проводящей полуплоскости и идеально проводящего клина с произвольным углом раствора в работах [90,91]. Подробно строгая теория дифракционного излучения на ряде объектов представлена в обзорах Б. М. Болотовского с соавторами [92,93]. Решения многих задач, основанных на приближенным методах расчёта дифракционного излучения, представлены в работах А. П. Потылицина, М. И. Рязанова, М. Н. Стриханова и А. А. Тищенко [94–96], а также в их монографии [97].

Основное применение дифракционному излучению, как и переходному, было найдено в области детектирования частиц и диагностики пучков [98–102]. Его явным преимуществом по сравнению с переходным излучением является то, что траектория заряда не пересекается с неоднородностью. Этот факт позволяет разрабатывать маловозмущающие методики диагностики пучков.

Отдельно нужно отметить особый вид дифракционного излучения, возникающего при движении заряда вдоль периодической структуры (дифракционной решётки), которое носит название излучения Смита-Парселла в честь учёных, впервые наблюдавших его [103] (интересно, что И. М. Франк обратил внимание на этот эффект [104] ещё до открытия переходного излучения). Данному типу излучения посвящено большое количество как теоретических, так и экспериментальных работ.

Искусственные периодические структуры анализируются обычно именно ради получения излучения Смита-Парселла. Это излучение, как правило, имеет длины волн, сопоставимые с периодом структуры или меньше него. Медленные заряды могут, в принципе, генерировать излучение Смита-Парселла с большими длинами волн, однако настоящее исследование направлено, в основном, на рассмотрение задач с релятивистским движением пучков. К тому же характер самих рассматриваемых структур таков, что излучение Смита-Парселла на них возбуждается слабо.

Настоящая работа посвящена анализу процессов излучения релятивистских пучков частиц, характерные масштабы изменения (длины волн) которых велики по сравнению с периодом структуры. Роль последней играет либо планарная структура из параллельных проводников (неограниченная или полуограниченная), либо объемный метаматериал. Двумерная периодическая структура заменяется «экраном», на котором задаются «эффективные» («усредненные») граничные условия. Объёмная периодическая структура заменяется некоторой «метасредой» с определёнными «эффективными» параметрами. Возбуждаемое излучение можно при этом трактовать как излучение Вавилова-Черенкова, переходное или дифракционное излучение, в зависимости от геометрии конкретной задачи.

\hfill

\textbf{Актуальность темы.} Несмотря на значительное количество публикаций по процессам излучения пучков частиц, многие вопросы в этой области остаются недостаточно освещёнными. В определённой степени это обусловлено тем, что со временем появляется все больше новых материалов с необычными электродинамическими свойствами, примером которых могут служить «метаматериалы». Немалый интерес представляет и исследование излучения, порождаемого пучками заряженных частиц в присутствии достаточно известных структур, в частности, планарных систем из протяжённых проводников с малым периодом, которые также рассматриваются в данной диссертации. Ранее, как было отмечено, излучение в присутствии таких структур частично исследовалось, однако без детального анализа наиболее интересной его части — нерасходящихся поверхностных волн. Кроме того, все ранее проведённые исследования подобных задач ограничивались анализом излучения от точечного заряда, конечность размеров пучков частиц не принималась во внимание. Отмеченные выше факторы указывают на актуальность темы настоящей работы для развития фундаментальных представлений о процессах излучения зарядов.

Тема работы актуальна и для потенциальных приложений. Это обусловлено, прежде всего, тем, что рассматриваемые структуры позволяют генерировать практически нерасходящееся излучение (в виде либо объёмных, либо поверхностных волн). Наличие такого излучения позволяет рассчитывать на то, что его можно применить для развития метода диагностики пучков частиц, в том числе для определения их размеров и формы. Особый интерес представляют такие ситуации (которые реализуются в моделях, рассматриваемых в диссертации), когда пучок не разрушается ни за счёт непосредственного взаимодействия с элементами структуры, ни за счёт воздействия на него тормозящей и отклоняющей сил со стороны генерируемого поля излучения. В таких случаях можно надеяться на реализацию неразрушающего непрерывного мониторинга характеристик пучков.

\hfill

\textbf{Целью} работы является аналитическое и численное изучение волновых полей неточечных пучков заряженных частиц, которые движутся с постоянной скоростью в присутствии планарных и объёмных периодических структур из параллельных проводников. При этом рассматривается только относительно низкочастотное поле излучения, характерные длины волн которого велики по сравнению с периодом структуры. В такой ситуации двумерные структуры могут описываться с помощью «усреднённых» граничных условий, а объёмные — с помощью «эффективного» тензора диэлектрической проницаемости. Наиболее важной для данного исследования является нерасходящаяся часть генерируемого волнового поля, которая представлена поверхностными волнами в случае планарных структур и объёмным излучением в случае трёхмерного метаматериала.

В работе анализируются электромагнитные поля пучков, которые движутся с постоянной скоростью, имеют конечную длину и пренебрежимо малое поперечное сечение. Рассматриваются следующие случаи движения пучков:
\begin{itemize}
\item вдоль безграничной планарной структуры из параллельных проводников перпендикулярно им;
\item вдоль края полубесконечной планарной структуры из параллельных проводников перпендикулярно им;
\item сквозь безграничную планарную структуру из параллельных проводников перпендикулярно ей;
\item мимо края полубесконечной планарной структуры из параллельных проводников перпендикулярно ей;
\item внутри неограниченной трёхмерной структуры из параллельных проводников перпендикулярно им;
\item вдоль границы полубесконечной трёхмерной структуры из параллельных проводников перпендикулярно им.
\end{itemize} 

\hfill

\textbf{Краткое содержание диссертации}

\textbf{Первая глава} посвящена излучению пучков заряженных частиц, движущихся в присутствии планарной периодической структуры («сетки») из параллельных проводников. Период структуры считается малым по сравнению с длинами волн в рассматриваемой части спектра. Кроме того, предполагается, что толщина проводников мала по сравнению с периодом структуры.

В разделе 1.1. рассматривается модель, используемая для описания электромагнитных свойств планарной периодической структуры из параллельных проводников. Модель основана на известном методе усреднённых граничных условий (УГрУ). Согласно этому методу, сетка заменяется сплошной плоскостью («экраном»), на которой ставятся УГрУ. Указаны параметры, описывающие структуру в рамках этого приближения.

В разделе 1.2. рассматривается поле тонкого (т.е. обладающего малым поперечным размером) пучка заряженных частиц, движущегося равномерно и прямолинейно вдоль поверхности бесконечной планарной структуры из проводников перпендикулярно им. Изначально предполагается, что пучок имеет произвольный продольный профиль (продольную плотность распределения заряда). Методами теории функций комплексного переменного (ТФКП) анализируется поведение полного электромагнитного поля в дальней зоне. Показывается, что волновое поле состоит лишь из высоконаправленного излучения — двух симметричных друг другу поверхностных волн, которые сосредоточены вблизи плоскости структуры и распространяются вдоль проводников от линии движения пучка. В случае пренебрежимо малых потерь в проводах они не убывают со временем вдоль направления распространения. Подобные свойства обеспечивают сохранение пространственной структуры поля поверхностной волны. Показано, что её анализ может быть использован для определения размеров пучка заряженных частиц, порождающего волну. Приводятся аналитические выражения для поля поверхностной волны от пучков различных форм (точечный заряд, пучок с «прямоугольным» и гауссовым продольным распределением заряда). Рассчитываются также энергетические потери пучка на единицу длины пути.

В разделе 1.3. анализируется задача, схожая с рассмотренной в разделе 1.1, с той разницей, что вместо бесконечной планарной структуры берётся полубесконечная. При этом граница (край) структуры перпендикулярна проводам и траектории движения пучка. Решение ищется с помощью метода Винера-Хопфа-Фока. Показано, что волновое поле также состоит лишь из поверхностных волн. Однако их число и характер зависят от взаимного расположения пучка заряженных частиц и края структуры. Если проекция траектории пучка на плоскость, содержащую сетку, не попадает на саму сетку, то возбуждается лишь одна поверхностная волна, которая распространяется вдоль проводников без убывания (при отсутствии потерь в проводах), однако её амплитуда уменьшается с увеличением расстояния от пучка до края сетки. Если проекция траектории пучка на плоскость, содержащую сетку, попадает непосредственно на саму сетку, то возбуждаются четыре поверхностные волны: одна аналогична той, которая возбуждается в предыдущем случае, вторая и третья аналогичны волнам, возбуждаемым над бесконечной сеткой, а четвертая волна формируется в результате отражения второй от края сетки.

В разделе 1.4. рассматривается случай движения тонкого пучка заряженных частиц сквозь планарную структуру из параллельных проводников перпендикулярно ей. Из полного поля выделяются вклады, которое представляет объёмное излучение, имеющее относительно широкую диаграмму направленности, и поверхностную волну (высоконаправленная часть излучения). Отмечаются характерные особенности объёмного излучения: его отсутствие в плоскости, перпендикулярной проводам, возможность наличия двух локальных максимумов на диаграмме направленности и другие. Отмечаются свойства поверхностных волн, которые распространяются вдоль проводов от точки влёта пучка без убывания (в случае отсутствия потерь в проводах). Приводятся аналитические выражения для поля поверхностной волны от пучков различных форм.

В разделе 1.5. рассматривается электромагнитное поле тонкого пучка заряженных частиц пролетающего мимо края ограниченной планарной структуры из параллельных проводников перпендикулярно ей. Проводники располагаются в полуплоскости, а край структуры ортогонален им. Решение ищется с помощью метода Винера-Хопфа-Фока. Приводятся вклады объёмного излучения и поверхностной волны. Отмечаются особенности объёмного излучения по сравнению со случаем пролёта сквозь бесконечную структуру (в частности, это асимметрия диаграмм направленности и наличие излучения в плоскости, ортогональной проводникам). Далее проводится анализ структуры поверхностной волны, и отмечаются её особенности по сравнению со случаем неограниченной структуры.

Во всех разделах главы 1 приводятся результаты численных расчётов, характеризующие поле излучения. Среди них трёхмерные (цветовые) графики для компонент поля и плотности потока энергии поверхностных волн, диаграммы направленности объёмного излучения и другие.

\textbf{Вторая глава} посвящена анализу излучения тонких пучков заряженных частиц, движущихся в присутствии трёхмерной периодической структуры из длинных параллельных проводников — так называемого «проволочного метаматериала», описываемого с помощью «эффективного» тензора диэлектрической проницаемости.

В разделе 2.1. описаны макроэлектродинамические свойства данного метаматериала и особенности плоских волн в нем.

В разделе 2.2. рассматривается электромагнитное поле тонкого пучка заряженных частиц, движущегося внутри бесконечного проволочного метаматериала. Проводится строгое решение задачи в рамках рассматриваемой модели. Отмечается, что полное поле пучка разделяется на квазикулоновскую часть и поле излучения, проводится их анализ. Приводятся общие интегральные выражения для волнового поля пучка с произвольным продольным распределением заряда. Получаются точные аналитические выражения для волнового поля точечного заряда и «прямоугольного» пучка, движущихся с произвольной скоростью (они выражаются через известные спецфункции). Также приводятся точные аналитические выражения для квазикулоновского поля точечного заряда в ультрарелятивистском случае. Отмечаются необычные свойства излучения: оно распространяется вдоль проводов со скоростью света, концентрируется вдоль определённых линий позади пучка, не убывает со временем, не имеет порога по скорости пучка. Показывается, что пространственная структура поля волны не изменяется в ходе её распространения, а её анализ может дать информацию о размерах и форме пучка. Рассчитываются потери энергии пучка на излучение на единицу длины пути (тормозящая сила). Показывается, что в типичных условиях потери, как правило, будут несущественны по сравнению с кинетической энергией пучка. Приводятся примеры расчётов компонент поля излучения и плотности потока энергии для различных пучков.

В разделе 2.3. рассматривается поле тонкого пучка, движущегося вдоль плоской границы полубесконечного метаматериала, проводники которого ортогональны границе. Приводятся выражения для волнового поля в общем виде, а также в частных случаях «прямоугольного» пучка и точечного заряда. Продемонстрировано отличие поля излучения от случая бесконечного метаматериала. Оно выражается, в частности, в том, что излучение концентрируется вблизи некоторых линий, исходящих не из самого пучка, а из его проекции на границу метаматериала. Отмечается, что в случае движения пучка вдоль границы метаматериала амплитуда волнового поля убывает при удалении траектории пучка от границы. Отдельно рассматривается ситуация движения непосредственно по границе метаматериала, для которой приводятся точные аналитические выражения поля излучения, возбуждаемого точечным зарядом. Получены энергетические потери пучка на излучение на единице длины пути, а также отклоняющая сила, действующая на пучок. Приводятся графики, показывающие распределения поля излучения и плотности потока энергии.

В \textbf{заключении} представлены основные результаты, полученные в диссертации.

\textbf{Положения, выносимые на защиту. }
\begin{enumerate}
  \item В случае движения пучка заряженных частиц в присутствии безграничной или полуограниченной планарной периодической структуры из параллельных проводников перпендикулярно им показано следующее:
      \begin{itemize}
        \item[а)] нерасходящаяся часть поля излучения представляет собой набор поверхностных волн, которые возбуждаются при любой скорости пучка, распространяются вдоль проводов со скоростью света в вакууме, а пространственное распределение их полей не изменяется в процессе распространения;
        \item[б)] в случае полуограниченной структуры всегда возбуждается поверхностная волна, распространяющаяся вдоль проводов от края структуры; ее поле убывает с ростом расстояния от края до проекции пучка на плоскость, в которой расположена структура. Если пучок движется вдоль структуры, а его проекция попадает на полуплоскость, занятую проводами, то возбуждаются также другие поверхностные волны; одна из них отражается от края структуры, причем электрическое и магнитное поля отражённой волны равны по модулю полям падающей волны, но повернуты на определенный угол;
        \item[в)] сила радиационного торможения при движении пучка вдоль планарной структуры прямо пропорциональна его скорости, а также, в случае  пучка с постоянной продольной плотностью заряда, обратно пропорциональна линейной комбинации квадратов длины пучка и расстояния до сетки;
        \item[г)] если пучок движется перпендикулярно неограниченной структуре, то возбуждается также объёмное излучение, максимум которого лежит в плоскости, параллельной проводам и содержащей траекторию пучка; в направлении проводов и в направлении движения пучка излучение отсутствует. Если пучок движется мимо края полуограниченной структуры перпендикулярно ей, то диаграмма направленности излучения асимметрична относительно траектории пучка; при этом имеется излучение в направлении движения пучка.
      \end{itemize}
  \item В случае движения пучка заряженных частиц в бесконечном проволочном метаматериале или вдоль его границы перпендикулярно проводам показано следующее:
      \begin{itemize}
        \item[а)] поле излучения является беспороговым по скорости пучка, распространяется вдоль проводов со скоростью света, концентрируется вблизи определённых прямых линий, исходящих из пучка и составляющих тупой угол с его скоростью. Пространственная структура поля не изменяется в процессе распространения;
        \item[б)] при движении пучка в вакууме вдоль границы полубесконечного метаматериала волновое поле в нем концентрируется в окрестности определенных линий, исходящих из проекции пучка на границу;
        \item[в)] сила радиационного торможения при движении пучка внутри метаматериала прямо пропорциональна его скорости. В случае движения пучка вдоль полуограниченного метаматериала имеется отклоняющая пучок сила, которая в ультрарелятивистском пределе в два раза больше, чем тормозящая.
      \end{itemize}
  \item Аналитические и численные результаты во всех рассмотренных задачах показывают, что генерируемое возмущение, как поверхностное, так и объёмное, отражает размеры и форму пучков частиц. При движении пучка вдоль плоской структуры или вдоль границы объёмного метаматериала степень «размытости» получаемой картины волнового поля увеличивается с ростом расстояния от пучка до границы структуры.
\end{enumerate}

\hfill

\textbf{Научная новизна} представленных в диссертации результатов заключается в следующем.

Получены и исследованы аналитические выражения для электромагнитного поля пучка заряженных частиц с произвольным продольным распределением плотности заряда при его движении вдоль бесконечной или полубесконечной планарной структуры из параллельных проводников. Описаны отличия поверхностных волн в задачах с неограниченной и полуограниченной структурами. Описан эффект отражения поверхностной волны от края структуры. Получены потери энергии пучка на излучение при движении вдоль бесконечной структуры.

Получены и исследованы аналитические выражения для электромагнитного поля пучка с произвольным продольным распределением плотности заряда при его движении перпендикулярно бесконечной или полубесконечной планарной структуре из параллельных проводников. Описаны отличия поверхностных волн и объёмного излучения в задачах с неограниченной и полуограниченной планарными структурами.

Получены и исследованы аналитические выражения для электромагнитного поля пучка с произвольным продольным распределением плотности заряда при его движении внутри неограниченного проволочного метаматериала или вдоль границы такого метаматериала. Описаны свойства поля излучения. Продемонстрировано, что излучение является нерасходящимся и беспороговым по скорости. Аналитически получены потери энергии пучка на излучение, а также величина отклоняющей силы при движении пучка вдоль границы метаматериала.

Во всех рассмотренных задачах проведены численные расчёты полей излучения. Продемонстрирована возможность использования полей поверхностных волн (на планарных структурах) или объёмного излучения (в проволочном метаматериале) для определения размеров и формы пучка.

\hfill

\textbf{Научная значимость} полученных результатов заключается в аналитическом и численном исследовании процессов излучения пучков частиц конечного размера в описанных выше условиях.

В этой связи можно выделить:

\begin{itemize}
  \item описание поверхностных волн на планарных периодических структурах, как неограниченных, так и имеющих край, при движении пучка вдоль структуры и перпендикулярно ей;
  \item описание нерасходящегося беспорогового излучения Вавилова-Черенкова в неограниченном или полуограниченном проволочном метаматериале;
  \item получение выражений для потерь энергии пучка в ряде рассмотренных ситуаций;
  \item демонстрацию характерных особенностей излучения с помощью ряда численных расчётов как для планарных, так и для объёмных структур.
\end{itemize}

\hfill

\textbf{Практическая значимость} полученных результатов обусловлена возможностью развития нового метода диагностики пучков заряженных частиц при помощи периодических проволочных структур. Полученные в работе результаты показывают, что как планарные, так и объёмные системы из протяжённых проводников позволяют генерировать поля излучения, пространственные распределения которых содержат информацию о длине пучка. Более того, по объёмному излучению в метаматериале можно определять и поперечные размеры пучка. Преимуществом данного метода является то, что он является неразрушающим по отношению к пучку. Кроме того можно отметить, что полученные в работе результаты могут быть использованы для расчётов полей зарядов произвольной формы с помощью алгоритмов, основанных на аналитическом решении уравнений Максвелла, без затраты большого количества вычислительных ресурсов.

\textbf{Достоверность} полученных результатов обеспечивается последовательным применением строгих аналитических методов, разработанных в электродинамике, теории функции комплексного переменного и математической физике, а также совпадением результатов диссертации с известными из литературы частными случаями.

\textbf{Личный вклад автора.} Содержание диссертации отражает персональный вклад автора в проведённые исследования. Во всех случаях автор диссертации принимал активное участие в постановке задач, выборе методов исследования и анализе полученных результатов (совместно с научным руководителем). Вклад автора в процесс получения всех основных результатов диссертации был определяющим.

\textbf{Публикации.} По результатам диссертации опубликовано 14 работ [105–118]. Основные научные результаты отражены в 5 статях в международных журналах, входящих в базы данных «Web of Science» и «Scopus» [105–109]. Результаты исследований излагались также в ряде полнотекстовых докладов [110–115] и тезисов докладов [116–118] ведущих в своей области международных и российских конференций.

\textbf{Апробация работы.} Доклады по результатам, описанным в диссертации, представлялись на XV Всероссийской научной конференции студентов-радиофизиков (Санкт-Петербург, 2011) [114], Региональной XX конференции по распространению радиоволн (Санкт-Петербург, 2014) [115], ведущей ежегодной международной конференции по физике ускорителей «International Particle Accelerator Conference» в 2012–2015 годах (Новый Орлеан, США, 2012; Шанхай, Китай, 2013; Дрезден, Германия, 2014; Ричмонд, США, 2015) [110–113], международной конференции «Days on Diffraction» в 2012 и 2013 годах (Санкт-Петербург) [116, 117], международном семинаре «IV Mini-workshop for Advanced Generation of THz and Compton X-ray Beams using compact electron accelerators» (Санкт-Петербург, Россия, 2014), а также на IX и XI международных симпозиумах «Radiation from Relativistic Electrons in Periodic Structures» (Лондон, Великобритания, 2011; Санкт-Петербург, Россия, 2015) [106,118]. Большинство докладов было представлено автором лично.

\textbf{Объём и структура диссертации.} Диссертация состоит из введения, двух глав и заключения. Полный объем диссертации составляет 126 страниц, включая 48 рисунков и список литературы, содержащий 145 наименований.

