\chapter{Излучение пучка заряженных частиц в присутствии планарных периодических структур из тонких параллельных проводников} \label{ch:1}

Данная глава посвящена исследованию излучения пучков заряженных частиц в присутствии планарных периодических структур, состоящих из тонких длинных параллельных проводников. Рассматриваются четыре задачи с различной геометрией: в двух из них периодическая структура считается неограниченной, а в двух других — полуограниченной (с проводниками, ортогональными её краю). Сначала анализируется поле, создаваемое пучком заряженных частиц, который движется на некотором расстоянии параллельно бесконечной структуре и перпендикулярно проводам. Затем рассматривается полубесконечная структура, которая возбуждается пучком, движущимся вдоль её края. Далее изучается случай пучка, пролетающего сквозь бесконечную структуру нормально к её поверхности, а после этого анализируется случай движения мимо края полубесконечной структуры в ортогональном ей направлении.

В работе исследуется «длинноволновая» часть в спектре излучения, для которой характерные масштабы изменения значительно превышают период структуры. Данное предположение позволяет воспользоваться для описания воздействия структуры на поле пучка так называемыми усреднёнными граничными условиями (УГрУ).


\input{Dissertation/sec1_1}